\documentclass[10pt,a4paper,oneside]{article}
\usepackage[utf8]{inputenc}
\usepackage[english]{babel}
\usepackage{amsmath}
\usepackage{amsfonts}
\usepackage{amssymb}
\usepackage{graphicx}
\usepackage{url}
\usepackage{anysize}
\usepackage{cite}
\usepackage[hidelinks]{hyperref}

\renewcommand{\familydefault}{\sfdefault}
\marginsize{3.5cm}{2.5cm}{1.5cm}{1.5cm}

\setlength{\parindent}{0pt}
\linespread {1.25}


\author{Tobias Meier (meierto3), Yacine Mekesser (mekesyac)}
\title{\large{Abstract}}
\begin{document}
	
	\makeatletter
	\renewcommand{\maketitle}{\bgroup\setlength{\parindent}{0pt}
		\begin{flushleft}
			\textbf{\@title}
		\end{flushleft}\egroup
	}
	\makeatother
	\thispagestyle{empty}
	
	\maketitle
	The rising complexity of software systems and agile development methods makes it increasingly difficult to control the quality of a software project. This makes a system for defect prediction desirable.
	We assume that machine learning offers the potential to realise  such a solution. The goal of this bachelor thesis is to expand existing approaches by incorporating concepts originating in text analysis, especially N-Grams. Furthermore, a foundation for future work on this topic should be created.
	For this, a comprehensive and modular toolset was developed. It is able to analyse the Git repository of arbitrary Java projects. The extracted data can then be used as the basis for training a machine learning algorithm. With the resulting model, we try to predict how many bugfixes a file version will receive in the coming months.
	The implemented solution shows a significant correlation between the features used and the error-proneness of Java files. However, the results of our experiments could not conclusively prove the usefulness of N-Grams in defect prediction.  	
	\\
	\\
	\paragraph{Keywords:} Defect prediction, Machine Learning, Regression, N-Grams, Repository Mining, Software Metrics, Feature Design

\end{document}